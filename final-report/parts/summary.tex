%----------------------------------------------------------------------------------------
%	Executive Summary
%----------------------------------------------------------------------------------------

\chapter{Executive Summary}

The Society for the Prevention of Cruelty to Animals (SPCA) is an organisation that rescues animals with harsh lifestyles and aims to rehome them into better lives. As part of their rehoming process, the SPCA aims to improve and maintain good health for their animals.

The weight of an animal tends to be a vital indicator of its overall health. Over the past three months, we have proposed and developed a product that makes the weighing and data management process of these animals as easy as possible for the organisation's members, with our current prototype being aimed toward dogs in particular. Our product will allow SPCA staff members to weigh any dog between 0 to 25kg with approximately 0.2kg precision, upload the data automatically to a server for storage, and automatically calculates their weight trends and metrics. The product comprises two parts: a physical scale that weighs the animals and processes their data, and a service that stores and secures the weight data, allowing users to interact with other members as well as the data and transfer it in multiple ways. A team of specialists has combined these two parts to make our product work as seamlessly as possible, allowing users to interact with the entire product easily and without hassle. During the development of our product, we considered many societal factors, namely Te Tiriti o Waitangi obligations, data privacy and security, and RoHS and WEEE compliance. 

The simplest and most common use case of our product is as follows: the staff member will log in, and then select the dog that they wish to measure. They will select the ‘+’ button in the software to add a new weight, then use the scale to weigh the dog. As well as the weight being displayed via the seven-segment display, the weight will also be automatically sent to the software, where it will be saved by the user.

The next step to bring our product into use at the SPCA is to consult with members of the SPCA to ensure that we have met all of their requirements, and from there improve our design based on their feedback. Then the scale would be manufactured en masse, the data would be transferred to a long-term server, and the software applications would be hosted for easy access. This would allow our product to be utilised by SPCA staff on a daily basis. The estimated cost of mass production is \$3500 including hardware and certification, then the ongoing cost for the software is \$50 per month.

In the future, we would like to make our product more accessible by expanding the service to work on other operating systems, such as iOS, expanding the utility by allowing more animals to be weighed, adjusting the physical scale for ease of use, and improving the quality of components used to build the scale for robustness such as electrical components, physical interfaces (display, buttons, LEDs), and microcontrollers. 

