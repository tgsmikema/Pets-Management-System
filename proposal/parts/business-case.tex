%----------------------------------------------------------------------------------------
%	Business Case: Introduction
%----------------------------------------------------------------------------------------

\chapter{Overview}

TODO: Introduce the project and what it will deliver

\chapter{Deliverables}

We will implement our prototype over the following semester, which will include building the scales, linking it to the software using embedded software, and creating a software backend and two frontends. To ensure that every team member contributes equally and all of the work is completed on time, we have created a Gantt chart to plan our development.  Please note that all team members have been anonymised to numbers.

INSERT GANTT CHART WHEN FINISHED

Team member 2 has the most experience in the electrical engineering space, so she will take the lead for all electronic components. She will be supported by member 6, and 3 and 7 both have the knowledge required to assist if necessary. 3 will coordinate the effort on the embedded components, which will include 6 and 7. Member 1 will lead the software development process. This will be broken down into backend and frontend, with 1 and 4 working on the backend and then assisting 5 and 8 with the frontend when they are finished. Due to the large scope of the frontend, 3, 6 and 8 will also help to develop these applications as they have experience with the required technologies. Having members of the team shift around between sections will ensure that all parts of the project are coordinated and will work together once assembled. We will leverage the diverse experience of all team members to create the best possible product. 


\chapter{Requirements}

This application will help the SPCA to more efficiently and effectively manage the weight of their animals. For the purpose of a minimum viable product (MVP), we will only focus on dogs between 5 and 25kgs. Our application will help SPCA staff to more easily weigh animals by recording the weight automatically. It will also streamline the process of viewing a dog’s weight history, identifying trends, and noting any problems. In the future there is a possibility of expanding this to all animals and to encompass more health areas. 

Through our work in the electrical section, we will ensure that the weights are recorded quickly and accurately. Our custom-built scale will record the dogs weight electronically, and will display it through an LED display. The embedded software will allow this measurement to be sent using Wi-Fi to a staff member’s phone and/or computer. The software will record this weight, and keep track of a dog’s history. This will allow staff members to view this data and identify issues. The intuitive interface of the software will make it easy for staff to use, and will have a low learning curve to ensure that the system can be used as quickly as possible and with minimal obstacles. 


\section{Te Tiriti o Waitangi}

TODO: How are the Te Tiriti o Waitangi goals met

\section{Sustaintability and Environmental}

TODO: How are the sustaintability goals met

\section{Legal and Privacy}

TODO: How are the legal and privacy goals met

\section{Other Ethical Issues}

TODO: How are the ethical goals met

