%----------------------------------------------------------------------------------------
%	Business Case: Introduction
%----------------------------------------------------------------------------------------

\chapter{Overview}

TODO: Introduce the project and what it will deliver

\chapter{Deliverables}

We will implement our prototype over the following semester, which will include building the scales, linking it to the software using embedded software, and creating a software backend and two frontends. To ensure that every team member contributes equally and all of the work is completed on time, we have created a Gantt chart to plan our development.  Please note that all team members have been anonymised to numbers.

INSERT GANTT CHART WHEN FINISHED

Team member 2 has the most experience in the electrical engineering space, so she will take the lead for all electronic components. She will be supported by member 6, and 3 and 7 both have the knowledge required to assist if necessary. 3 will coordinate the effort on the embedded components, which will include 6 and 7. Member 1 will lead the software development process. This will be broken down into backend and frontend, with 1 and 4 working on the backend and then assisting 5 and 8 with the frontend when they are finished. Due to the large scope of the frontend, 3, 6 and 8 will also help to develop these applications as they have experience with the required technologies. Having members of the team shift around between sections will ensure that all parts of the project are coordinated and will work together once assembled. We will leverage the diverse experience of all team members to create the best possible product. 


\chapter{Requirements}

This application will help the SPCA to more efficiently and effectively manage the weight of their animals. For the purpose of a minimum viable product (MVP), we will only focus on dogs between 5 and 25kgs. Our application will help SPCA staff to more easily weigh animals by recording the weight automatically. It will also streamline the process of viewing a dog’s weight history, identifying trends, and noting any problems. In the future there is a possibility of expanding this to all animals and to encompass more health areas. 

Through our work in the electrical section, we will ensure that the weights are recorded quickly and accurately. Our custom-built scale will record the dogs weight electronically, and will display it through an LED display. The embedded software will allow this measurement to be sent using Wi-Fi to a staff member’s phone and/or computer. The software will record this weight, and keep track of a dog’s history. This will allow staff members to view this data and identify issues. The intuitive interface of the software will make it easy for staff to use, and will have a low learning curve to ensure that the system can be used as quickly as possible and with minimal obstacles. 


\section{Te Tiriti o Waitangi}

In New Zealand, it is important that we fulfil our treaty obligations to make the system usable by different cultures including Māori. This means we must recognise and respect the treaty principles of partnership, participation, protection and options. We must also protect tino rangatiratanga of resources. For the purposes of this project these obligations are either in scope or out of scope. 

The principles of protection and tino rangatiratanga over resources are in scope for this project. The resources that we use and collect are mainly in the form of data. In Te Ao Māori, data are taonga and we must recognize tangata whenua as kaitiakitanga of their data. To fulfil this obligation we need processes in place to allow any user to maintain full control over their data including access, management, deletion. We will also use Māori place names in our frontend application to support the protection and revitalisation of  te reo Māori.   

Other obligations like partnership, participation and options are out of scope because we lack resources to meet and discuss with tangata whenua the development and implementation of our project. To ensure the system is usable by different cultures would require consultation with representatives of different cultures to understand their needs. An example of this could be offering our app and website in different languages. With no money or other resources to implement these changes they remain out of scope. 


\section{Sustaintability and Environmental}

Many of the sustainability concerns involve the components that we are using and their effects on the environment. WEEE (electrical and electronics waste) is increasing year by year so we need to reduce the amount of waste to the best of our abilities. Starting with the electrical components, PCBs are mostly made out of resin and metals including copper, tin and lead. All these materials are potentially hazardous if they are not disposed of properly.

In order to maximise sustainability, we need to ensure that we are designing our circuit in a way that uses the least amount of materials and dispose of them correctly after use. Other than this, we can choose to use components that comply with the RoHS restrictions and are less harmful to the environment. A plan for what happens to the hardware of the smart scale at its end of life should also be developed. For example, it could have instructions on it to be taken to a user’s local electronics disposal centre for responsible waste management. Further, a plan should be developed for when maintenance or repairs are required. For example, the team could have a repair policy where the scale can be sent back and refurbished in order to utilise components that remain functioning. However, this needs to be further evaluated as it would incur extra cost, and also require having technical staff available to interface with users and provide technical support. Thus, this idea is currently out of scope, but should be further investigated.

In terms of the Raspberry Pi, we have little flexibility in terms of sustainability, as the Pi is chosen for us. Therefore the Compsys team needs to make sure any additional components that we decide to use (like seven-segment displays) are recyclable and that we don’t use anything beyond what we need (choosing a display that’s larger than what we need). 


\section{Legal and Privacy}

TODO: How are the legal and privacy goals met

\section{Other Ethical Issues}

TODO: How are the ethical goals met

