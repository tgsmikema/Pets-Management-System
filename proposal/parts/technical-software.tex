%----------------------------------------------------------------------------------------
%	Software
%----------------------------------------------------------------------------------------

\chapter{Software Components}

The software requirements of this project include an embedded software component, a backend and two frontend components. The embedded system will communicate with the digital scale and the software backend. The backend will store data and communicate with the frontends, which will display the information to SPCA staff and allow them to enter weights. 

This section will review some of the existing products, and will then specify our design of the various software components.

\subsection{Prior Art}

\subsection{Embedded}
The first step for the embedded section starts with the communication between the PCB and the Raspberry Pi. We are measuring the voltage of the output signal generated by the instrumentation amplifier to give us an indication of the weight present on the scale. We will also measure the voltage value of the PCB ground and a separate ADC channel, this will allow us to measure the inherent offset in the Pi’s ADC. We can then write firmware to convert the ADC values into weights as the circuitry has been designed to vary the output voltage linearly with weight. 

The measured weight will be the average of multiple samples. The scale should stop measuring once the weight has stabilised within a certain range. Then it will take an average of the samples measured and take that as the weight of the dog. The specifics of the algorithm will be developed through practical testing on a working prototype. 

Once we have these voltages, we can send them to the backend of the website and app via HTTP or HTTPS. (write more on the wifi once we’ve done more stuff)
The workflow for the communication between the Raspberry Pi and the backend is that every time a dog is weighed, the measurements and a scale ID are sent to the backend.The backend will always store the most recent value sent from each scale, but at this stage it is not associated with any dog. Once the user is happy with measurement, they can take the most recent weight and link it with a dog id to store it in the dog database.

We have decided to include a seven-segment display into our design. This is done for the purpose of convenience and reliability. It increases convenience as the user can immediately read the weights. It also adds reliability as the scale would be able to function without wifi, which may occur for several reasons. A shift register is connected to the Raspberry Pi to store the segments that should be lit. The communication between the register and Pi includes several connections - one for shifting the register and one for the data. 

\subsection{Backend}

\subsection{Frontends}
