%----------------------------------------------------------------------------------------
%	Technical Overview
%----------------------------------------------------------------------------------------

\chapter{Technical Overview}

To produce the smart scale product, there are several components including the hardware of the scale, the embedded software, and software. The hardware obtains electrical signals corresponding to the amount of weight applied to the scale. Weight sensing occurs via load cells, which are electrical sensors that react to changes in applied weight. The team must develop circuitry to connect the load cells and obtain electrical signals from them. The signal from the hardware will be input to the Analog to Digital Converter (ADC) of the scale’s microcontroller for use in the embedded software. Then, the embedded software processes the signal obtained from the hardware and sends it to the software. Finally, the software will have two front-end interfaces for a user to interact with the scale, and a backend to store the measurements gained from the scale. The front-ends will also allow the user to view trends over time and export weight data in order to better care for the animals’ health.
