%----------------------------------------------------------------------------------------
%	Hardware
%----------------------------------------------------------------------------------------

\chapter{Hardware Components}

The hardware design involves obtaining electrical signals corresponding to the amount of weight applied to the scale. The output signals of the hardware will be input to the Analog to Digital Converter (ADC) of the Raspberry Pi Pico microcontroller for use in the embedded software. Weight sensing occurs via load cells, which are electrical sensors that react to changes in applied weight applied. The team must develop circuitry to connect the load cells and obtain electrical signals from them. However, load cells change resistance minimally with weight applied. Thus signals obtained from them are small and must be amplified to be measurable by an ADC. Therefore, circuitry is also required to process the signals from the sensing circuit.

This section details the hardware design, including research on prior solutions, working principles of load cells, sensing and processing circuits, and the proposed design.

\subsection{Prior Art}
In this section, existing hardware solutions for digital weighing scales are evaluated. A digital weighing scale is a common measurement device, thus there are many existing products on the market. Several existing options are detailed and compared in Table 1.

TABLE

Generally, all physical weight scales are the same, with differences being in the more expensive range having the ability to connect to other devices where weight metrics are processed with other body measurements to give users more information on their health. This principle can be applied to not only humans, but also dogs. Our focus is to provide the weights of dogs, but also to upload that information to a database where further processing may take place, or purely just to track weight.

\subsection{Load Cell Working Principles}
This section explains the working principles of the load cells. A strain gauge is a resistor that either increases or decreases in electrical resistance when weight is applied. Thus, it is an electrical sensor that allows weight to be measured. A load cell uses several strain gauges to measure weight more accurately than one strain gauge. The provided load cells consist of two strain gauges as in Figure 1, one’s resistance decreases linearly with weight, and the other’s resistance increases linearly. To connect the load cells to circuitry, three wires are accessible from each; one at the midpoint of the resistors and one at each of the two ends. Four load cells are provided, with one at each corner of the scale, referred to as R1, R2, R3, and R4.

\subsection{Sensing Topologies}
In this section, topologies to connect the load cells are compared.

The load cells could individually be connected to the ADC to gain a measurement from each, as in Figure A.1. This would allow weight distribution to be sensed across the scale by gaining measurements from each corner; however, it requires four signals to be input to the ADC. With only three ADC ports, this option is only possible if an external ADC is used at additional cost. An extra cost is incurred from requiring a processing circuit for each signal.

Another option is connecting all four load cells together in a parallel configuration as in Figure A.2 and sending a single signal to the ADC. This option is low cost as it only requires one processing circuit. However, parallel connections reduce resolution because each load cell's resistance changes are averaged instead of being added.

Alternatively, each load cell could be connected in a Wheatstone bridge as in Figure A.3.a. This topology is used to detect small changes in resistance and produces a high-resolution output. Having one Wheatstone bridge per load cell would allow the sensing of weight distribution, however, it is not feasible because of the insufficient number of ADC ports. Two load cells could be placed in one Wheatstone bridge for two outputs. This would allow weight distribution sensing across halves of the scale. However, this provides little benefit over a single output and incurs additional cost. As such, all four load cells can be connected in one Wheatstone bridge as in Figure A.3.b. Adding multiple load cells in the Wheatstone bridge does not reduce resolution because differences are added and not averaged. Further, this option only requires one processing circuit; thus is a low-cost solution. 

\subsection{Processing Topologies}
This section discusses requirements for the processing circuit and compares topologies to process the signals from the sensing circuit. Processing the signals involves applying a DC offset, amplifying the signal, and filtering out noise.

Processing the signal from the sensing circuit is required because the signal is small, as mentioned, and because the ADC of the Raspberry Pi Pico has requirements and limitations. Further, amplification will require an operational amplifier (op-amp), a common amplification device. However, low-cost op-amps are not able to output tiny signals. Thus, the requirements for the output signal are: the signal should be between 0 V and 3.3V for the ADC, the signal should maximise this range to maximise resolution, and an offset to the signal is required to stay within the bounds of the op-amp supply rails.

Firstly, to produce an offset a Zener regulator may be used to produce as in Figure A.4.a. However, the offset it produces is limited to standard values due to the Zener diode and thus is not a flexible solution. Alternatively, two resistors may be used as a voltage divider to divide the supply voltage down to any value. However, the circuitry it is added to may draw current from it, thus changing the offset value (the loading effect). Therefore, this solution is not reliable. An op-amp buffer may be placed after the voltage divider as in Figure A.4.b to prevent loading because it prevents current from being drawn from the resistors.

The signal must also be amplified from a small voltage from the sensing circuit, to a signal that maximises the range of the ADC. An inverting or non-inverting op-amp amplifier may be used. However, these solutions only accept single ended inputs (referenced to ground), and the Wheatstone bridge produces a double ended input, thus are not suitable. An amplifier that accepts double ended inputs is the differential amplifier as in Figure A.5. This amplifier will successfully apply the gain and offset, however is susceptible to the loading effect (as defined earlier). To prevent the loading effect, op-amp buffers can be placed before the differential amplifier. This is the basis of the instrumentation amplifier as in Figure A.6. The instrumentation amplifier also tends to more effectively filter out electrical noise.

In any electrical signal, unwanted high-frequency harmonics or noise is present. Noise in the hardware output signal will reduce the accuracy of the ADC reading of the signal. To minimise its effects, a filter should be implemented. A simple resistor capacitor (RC) filter  as in Figure A.7 will be used to keep the design simple and cheap. An op-amp (active) filter could be used; however, many active filter topologies are complex and unnecessary for a signal that varies slowly with time.

\subsection{The Proposed Design}
This section outlines the proposed design, including a schematic, simulation results, specifications, and an expected bill of materials.
The design comprises a Wheatstone bridge, instrumentation amplifier, and RC filter, as in Figure 2.

The design was simulated using LTSpice as in Figure 3. Values from practical testing were used for base resistance of resistors and the amount of change of resistance as weight is applied. It is observed that the output varies linearly, and is amplified and offset successfully.

Using the LM324 op-amp, the minimum output voltage is 20 mV, so an offset is applied to avoid signals getting clipped. Further, the amplifier gain was chosen to maximise the output range of the LM324 from 20 mV to 1.8 V. The Wheatstone bridge also has a 1 mV offset due to differences in base resistance thus with the gain applied, results in a nearly 800 mV offset.

Functional and technical specifications of the design are summarised.

Functional specifications:
\begin{itemize}
    \item Power supply:	4 AA batteries
    \item Operating voltage:	3.3 V
    \item Measurement range:	0 kg to 25 kg
    \item Measurement resolution: 	0.5 kg
    \item Method of zeroing scale:	Automatic upon turn-on and/or button
\end{itemize}

Technical specifications:
\begin{itemize}
    \item Number of load cells:	4
    \item Sensing circuit topology:	Single Wheatstone bridge
    \item Processing circuit topology:	Instrumentation amplifier
    \item Gain of processing circuit:	714 V/V
    \item Offset of processing circuit:	0.1 V
    \item Range of output:               	0.8 V - 1.8 V
\end{itemize}

The bill of materials is described in Table 2. Costs are gained in NZD from in-stock items at Digikey.co.nz for one scale. Some components listed are not in Figure 2 because they are required but not fundamental to sensing or processing.
